
\section{Conclusions\label{sec:Conclusions}}

The main focus in the thesis have been to investigate a model driven
approach to handling data manipulation and persistence in an object
oriented language. The goal has been to eliminate a series of issues
that often arise in projects involving object oriented languages,
when coupled with external database systems.

Using EDMA, a user is able to transform a conceptual model into usable
code, shortening the path between getting an idea for a data model,
and having a running implementation. This is done through three steps: 
\begin{itemize}
\item First the user must define a conceptual data model, and write it in
the created EDMA data definition language. 
\item Then he must define the transaction signatures on the data model. 
\item At last, the user must implement the transactions in Java.
\end{itemize}
A Java interface containing the defined transactions is created, and
can be used from any java application. The EDMA runtime system is
designed in a modular way, allowing for future optimization of the
persistence mechanism, the set manipulations and the runtime data
containment.
\begin{itemize}
\item transaction language would be interesting\end{itemize}

