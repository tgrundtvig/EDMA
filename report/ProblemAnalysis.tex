
\section{Problem Analysis}


\subsection{Identifying the problem}

An abstract or conceptual data model is often a good place to start,
when modeling objects and concepts from real or fantasy worlds. The
conceptual data model describes the different kinds of entities in
the model and the relations between them.

From our experience with programming we have noticed, that in the
process of moving from a conceptual data model in a specific domain
into an actual working implementation of the model, there are a number
of hurdles that slows down the process:
\begin{enumerate}
\item \textbf{The amount of code}\\
The amount of code it takes to implement a conceptual data model is
much larger than the description of the conceptual model. Even with
the help of copy-paste and modern IDEs with code templates and code
completion, it still takes quite some time to write the code.
\item \textbf{Coding of bi-directional entity-relationships}\\
In a conceptual or abstract data model an entity-relationship is a
relationship between two entity kinds. e.g. a student can be enrolled
on a course. This is a many-to-many relationship, meaning that there
can be many different students enrolled on the same course and a student
can be enrolled on many different courses.\\
This could be implemented by adding a set of students to the course
class. And create methods to add and remove students. But if you then
want the courses that a specific student is enrolled on, you would
have to go through all the courses and lists of students. Therefore
one would most often also add a set of courses to the student class,
and make sure that this set is updated when a student is added to
or removed from a course.
\item \textbf{Accessing instances of the model from remote locations}\\
Many software systems are distributed on several different machines
and thus needs to communicate over the boundaries of different VMs
and physical machines. This means that data must be serialized and
de-serialized preferably in a seamless manner. Even though there exists
tools like Java RMI that can automate some of this process, it still
takes time configure and use these tools.
\item \textbf{Providing user interfaces for data input}\\
Software systems often interact with human beings and must provide
interfaces that can give information to and get information from human
beings. Few programmers like to make user interfaces.
\item \textbf{Thread-safe access to the model in a multithreaded environment}\\
The programming of concurrent programs is always an interesting task.
Errors occurring from race conditions, dead locks etc. are often hard
to detect and can reside in the system undetected for a long time
before they suddenly surfaces and cause havoc.
\item \textbf{Persistence of the instances of the model (if needed)}\\
There are many ways to obtain persistence of data models from object
oriented languages. Most of them involves the installation and use
of some sort of database management system.
\end{enumerate}
Although all the mentioned hurdles can be overcome, we are lazy programmers
and we believe in what Terence Parr, the creator of ANTLR (Another
Tool For Language Recognition), has made his guidance principle: 
\begin{quotation}
Why program by hand in five days what you can spend five years of
your life automating?
\end{quotation}

\subsubsection{Restrictions}

This thesis is not about creating a fast and efficient database system
that can store huge amounts of data. There are already many good solutions
to this problem with decades of research and development on the market,
and we have no intentions of competing with those.


\subsection{Solving the problem}

We intend to solve the problem by building a software system that
should take as input:
\begin{itemize}
\item A definition of a conceptual data model.
\item A definition of an interface to the data model.
\end{itemize}
The system should then generate the following:
\begin{itemize}
\item Java Interfaces and classes that represents an object oriented representation
of the data model.
\item A java interface that reflects the defined interface to the data model.
\item Java stub classes where the user can implement the domain specific
methods of the generated data model interface, by using the generated
object oriented representation of the data model.
\item A server program and a client proxy that implements the interface
to the data model, so the instances of the data model can be accessed
seamlessly over a socket connection.
\item A text based user interface that lets a human user invoke the methods
in the interface to the data model.
\end{itemize}
When the user have implemented the interface to data model, he should
be able to create instances of the data model that obey the following
requirements:
\begin{itemize}
\item They must be thread safe. By this we mean that the instance should
act in every respect as if all the methods in the interface where
synchronized on the same object.
\item Upon the creation of a data model instance it should be possible to
choose to have the instance persisted. If this option is chosen, the
instance must be automatically persisted in a durable way, meaning
that when a call to a method in the data model interface returns successfully
the state known by the calling thread immediately before the return
should be persisted successfully.
\item The methods in the interface to the data model must be executed atomically.
By this we mean that when the user implements the methods, he should
be given the opportunity to automatically roll back at any point during
the execution, if the execution cannot be completed successfully.
This should then bring the data model instance back to the same state
it had upon the beginning of the execution.
\end{itemize}
We will call the system Effective Data Model Abstraction (EDMA).

We will develop the system using a modular interface based design,
so the individual parts of the system can easily be replaced with
alternative implementations.

Instead of auto-generating all the code that is needed to implement
a specific data model, we will create a general runtime interface
that are independent of the specific data models. We will then only
auto-generate interfaces and binding code that binds the generated
data model specific interfaces to the general runtime interface.

We will strive to design the runtime interface in a way that gives
the maximum possible flexibility in how different implementations
handle the defined requirements on the data model instances. In this
way we are using our best effort to abstract the data models away
from the underlying technology, thus decoupling the data model design
and business logic from the applied technology.

With this design we are still strongly bound to the Java language,
since the user must implement the data model interface in Java. One
way to get around this, would be to design a language for interacting
with a conceptual data model and then have the user implement the
interface in this language and translate it to various target languages.
But then we could not benefit from the highly specialized Java IDEs
that makes coding fun and enjoyable.


\paragraph{Data Values}

In a data model we have different kinds of entities, each kind of
entity defines some attributes that can hold values in an entity instance.
These values can be different in each entity instance, but they will
have the same meaning and type across all the entity instances. As
an example a kind of entity could be Person and have the attribute
name of type string. Then all Person instances would have a value
of type string. The values could be different in the instances but
they would all represent the name of the Person instance. It would
not make sense to store an email address or an URL in that value,
although the type system would allow it because they are all strings.
The type string refer to an underlying representation of the value
and is not a very abstract concept. Instead the name of a Person should
have the type PersonName and the email address should have the type
EmailAddress and it should not be possible to store a persons name
in an attribute with the type EmailAddress. This would be a much more
fine grained type system.

As a part of the EDMA System we will define a fine grained data type
system for arbitrary complex, but immutable data values, where the
user can define his own semantically unique data types with build-in
validation for a specific domain. We will call this the value domain
system. The value domain system will give us several advantages:
\begin{itemize}
\item It can be used as an infrastructure for moving data between different
applications and data model instances. In this way all mutable state
can be contained inside data model instances and everything outside
is immutable. This makes it simple to distribute data model instances
seamlessly in a distributed system.
\item It lets the java compiler find a new category of semantic errors for
example if an email is used in the place of a name, because even though
they are both strings, they will have different types in the value
domain system and thus be represented by different Java classes.
\item It adds useful domain specific semantic information to the interfaces
of the data models.
\item It adds useful domain specific semantic information to the meta model,
that a code generator can use to auto generate utility programs that
are specialized to work on a specific data model within a specific
domain.
\end{itemize}
We will try to keep the generated and the user written code as separated
as possible. When there is a need to embed user written code into
a generated file, that file should be kept to a minimum size, merely
functioning as a container for the user written code. We will also
strive to generate code that looks and feels like well-written, handmade
code including javadoc.


\subsection{Related Work}


\paragraph{Entity-Relationship model}

Our meta data model is heavily inspired by the entity-relationship
model (ER model). The ER model is an abstract and conceptual way of
representing a data model\cite{chen1976entity}. The main concepts
in the ER model are entities and the relationships between them. The
ER model strives to be a representation of the data model that is
close to the human perception of the data model and is not concerned
with how the data is physically stored.


\paragraph{Apache Torque}

Apache Torque is an object-relational mapper for Java, that also uses
the idea of auto-generating the object oriented model instead of having
the user write it by hand and use annotations for the mapping\cite{torque}.

In Torque the user defines the data model in an XML schema and Torque
then generates both the object oriented model, the database schemes
and the binding between the object oriented model and the relational
database. Torque can also generate the XML schema from an existing
database model.


\paragraph*{Eclipse Modeling Framework}

Some inspiration for EDMA came from working with the Eclipse Modeling
Framework (EMF) in a model driven develop course. The Eclipse Modeling
Framework is a huge framework with a lot of useful functionality.
Unfortunately we did not feel that it was very pleasant to work with.
As an example, data models in the eclipse framework are defined in
an XML-based language called XMI. For a simple \emph{book} class with
a \emph{title} attribute and a \emph{pages} attribute the definition
looks like this:

\begin{lstlisting}[basicstyle={\footnotesize},breaklines=true]
</xsd:schema><ecore:EPackage xmi:version="2.0" xmlns:xmi="http://www.omg.org/XMI"
  xmlns:xsi="http://www.w3.org/2001/XMLSchema-instance"
  xmlns:ecore="http://www.eclipse.org/emf/2002/Ecore"
  name="library "nsURI="http:///library.ecore" nsPrefix="library">
<eClassifiers xsi:type="ecore:EClass" name="Book">
  <eStructuralFeatures xsi:type="ecore:EAttribute" name="title"
      eType="ecore:EDataType http://www.eclipse.org/emf/2002/Ecore#//EString"/>
  <eStructuralFeatures xsi:type="ecore:EAttribute" name="pages"
      eType="ecore:EDataType http://www.eclipse.org/emf/2002/Ecore#//EInt"/>
</eClassifiers>
</ecore:EPackage>
\end{lstlisting}


It is also possible to define data models through commercial graphical
modeling tools or annotated java files.

The EMF auto-generates a lot of Java code that the user is supposed
to make changes to or inherit from. This quickly becomes very large
files with mixed usercode and generated code where one can easily
get lost.
