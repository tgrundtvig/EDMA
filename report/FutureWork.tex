
\section{Future Work}

There are a number of features that could still be implemented in
EDMA, especially related to usability. By trying to use EDMA in various
situations, we have found that the time the user spends on reaching
a readily usable data model can be reduced. For example, we could
auto-generate CRUD actions and views (methods for Creating, Reading,
Updating and Deleting).
\begin{description}
\item [{CRUD~Actions~and~Views}] One way of helping the user, is to
auto-generate actions and methods for creating, reading, updating
and deleting (CRUD) data entities. At first, it was our philosophy
that the user should only have the actions and views defined by himself,
in order to not clutter the API with functions that wouldn't be used.
However, by creating data models with our system ourselves, we have
found that we almost always end up creating the most basic actions
and views for each kind:\end{description}
\begin{itemize}
\item Insert a new entity of a given kind.
\item Get entities based on attribute values (defined from indexes).
\item Update one of the mutable attributes in a kind.
\item Delete an entity.
\end{itemize}
As of the current version of EDMA, the user has to define these 4
methods in an \texttt{.edma}-file, and implement them in the generated
Java stub-code. Further more, for each relation, the user often needs
to connect two or more entities, or to delete a connection between
two or more entities.

It would require two changes to the system, to auto-generate CRUD
and relation-methods:
\begin{itemize}
\item Changing the compiler to add meta-actions and meta-views, for each
kind and each relation.
\item Changing the generator to generate the implementations of the actions
and views, instead of having the user to fill in stub-code in the
auto-generated actions and views.
\end{itemize}
Both of these two tasks are trivial, and could be implemented in a
short time. It would spare the user for the tasks of implementing
tedious, simple actions and views, at the cost of putting possibly
unused functions in the generated data model API.
\begin{description}
\item [{Multiple~APIs}] In a setting with many users, it is often possible
to distinguish between access rights of users. An extension to the
data model would be to group actions and views into interfaces. For
example, an interface for an Admin user could contain one set of actions
and views, while an interface for a User could contain a different
set of actions and views.
\item [{Simplifying~the~Value~Domain~System}] The value domain system
can be made more simple, by removing some of the primitive types.
For example, the types Integer and Long could be substituted with
a single Number type, and the Float and Double types could be substituted
with a single Decimal type. By inspecting range constraints it is
possible to determine best the representation on the underlying system
(in our case Java.)\end{description}

