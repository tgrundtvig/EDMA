
\section{\label{sec:Evaluation}Evaluation}

In this section, we address the issue of evaluating the EDMA system,
both with respect to its functional usability, and with respect to
a few performance parameters. Since one of the main points with EDMA
is to make it easy for the user to come from an idea of a data model,
to working implemented code reflecting that model, we briefly discuss
the steps that are required for doing so.

Although performance hasn't been the main focus of this project, we
have created a few simple performance tests. These are presented and
briefly discussed in Appendix \ref{sec:PerformanceEvaluation}.


\paragraph{Functional Evaluation}

Evaluating the usability of EDMA can be done by asking a series of
questions:
\begin{itemize}
\item How easy or difficult is it to create a new data model?
\item How easy or difficult is it to change an existing data model?
\item How easy or difficult is it to implement actions and views?
\end{itemize}
One way of measuring the complexity of using EDMA would be to get
a group of people to read the tutorial, and then carry out a list
of predefined tasks. The time it would take to complete each task
would then depict the complexity of the various parts of the system.
Another way of measuring the complexity, is to count the number of
steps it takes to optimally solve a certain task.

In answering these questions, we need a standard measure, to describe
how the tasks are carried out. As a measure, we count the number of
steps involved, and comment on the complexity of the steps.

The flow of creating a data model implementation can be divided into
4 phases:
\begin{itemize}
\item 1 -- Model definition
\item 2 -- Code generation (run the EDMA compiler)
\item 3 -- action and view implementation (create the necessary code in
the action and view stub classes)
\item 4 -- data model instantiation (create an instance of the data model
to use in the external application)
\end{itemize}
We see designing and implementing a concrete data model as an iterative
process, where the user switches between phase 1, 2 and 3 in a number
of iterations. In a prototyping scenario, where not all business requirements
may be strictly given, new requirements or ideas may influence and
change the design for a number of iterations. The user might work
in a typical prototyping scenario, going in and out of different work
phases in an undetermined order:
\begin{itemize}
\item Recognition and definition of the domain specific value domains
\item Recognition of the need for modeling a certain kind, relation, singleton,
action or view
\item Creation or refining of the data and interface definition files
\item Generation of code and implementation of actions and views
\end{itemize}
In many prototyping scenarios, the user will not be able to create
the full data model in one iteration. Therefore, it is important that
the user can easily go back and update the definition files, re-generate
the code, and fill out stub code of newly created actions and views.
And if the latest change to a definition file has resulted in broken
code, it must be easy to find and correct the files in question. In
the following, we discuss the four steps that make up the process
of using EDMA.


\paragraph{Data Definition}

Data definition is done by creating one, or many, data definition
files, adhering to the syntax of the EDMA definition language. The
definition of the kinds will itself result in a number of iterations,
where the user realizes the need for value domains, relations and
actions and views. For example, when creating the Person, the user
realizes the need for the attribute value domains Name, Date, PhoneNumber
and Balance. The following is noticed.
\begin{itemize}
\item Each value domain and each relation takes up one line of code. Each
kind takes up one line for the declaration, plus one line for each
attribute. Actions and views take up one line for the declaration,
one line for the description, one for the error code, and one line
for each input and output.
\item The number of relations will roughly be following the number of relations.
\item The number of actions will often be around twice the number of kinds
and relations (the user can have creation and deletion actions for
each kind and for each relation, and additional views and actions.)
\item Creating the value domains may be time consuming the first time it
is done, but as the user creates subsequent data models, he will find
out that there is a common set of value domains that are needed in
many data models. Therefore, the user might create a file for holding
the basic value domains needed (Year, Month, Day, Date, Hour, Minute,
Second, Time, PosInt, NegInt, etc.)
\end{itemize}
Because of the number, and line-span, of actions and views, we estimate
that writing the action and views is the most time consuming process,
when writing the edma-files.


\paragraph{Running the EDMA Compiler}

Running the EDMA compiler is a question of running a Java program.
If there are errors in any of the edma-files, the compiler gives the
user a message stating the error type, and the location of the error.
After the compiler has been run, the user should make sure that the
IDE shows the newest version of the directory listing (in Eclipse,
this is done by right clicking on the project folder, and choosing
Refresh.)


\paragraph{Implementing Actions and Views}

The stub code that is generated for actions and views purposefully
contains a syntax error. In that way, the user can quickly find the
action and view stub classes that need to be filled out

If the user has already implemented some actions and views, and later
decide to recompile the edma-files, the existing actions and views
will be kept.


\paragraph{Instantiation}

Instantiating the created data model is done in very few steps, by
calling a static method to retrieve a meta model instance, and creating
a runtime factory by which to obtain a model instance.


\subsubsection{Evaluation of Tutorial}

One way of assessing the usability of the EDMA system, is to let a
user complete the tutorial (Appendix \ref{sec:EDMAtutorial}), and
note down the time it takes to complete the individual steps. 
